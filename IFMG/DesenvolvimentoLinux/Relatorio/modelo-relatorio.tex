\documentclass[10pt,a4paper]{report}

\usepackage[brazil]{babel}
\usepackage[utf8]{inputenc}
\usepackage[T1]{fontenc}
\usepackage{graphics,color}
\usepackage[lmargin=3cm,rmargin=2.5cm,tmargin=2cm,bmargin=2.5cm]{geometry}

\newcommand{\bola}{\resizebox{12pt}{!}{$\bullet$\ }}
\newcommand{\relatorio}[3]{\vspace{3mm}\noindent\bola\textbf{Relatório Semana - #1 a #2 - Postado em #3}}
\newcommand{\atividades}{\textbf{Estudos e atividades realizadas:\ }}
\newcommand{\aprendizado}{\textbf{O que aprendi nesta semana:\ }}
\newcommand{\duvidas}{\textbf{Dificuldades e dúvidas que encontrei:\ }}
\newcommand{\acoes}{\textbf{Ações e atitudes para resolver os problemas:\ }}
 
\begin{document}


\begin{center} \LARGE
   \textbf{Modelo Relatório} \\[10mm]
\end{center}

\bgroup \large 
\noindent
\textbf{Disciplina:} Desenvolvimento Linux \\
\textbf{Aluno:} Lucas Mateus Fernandes\\
\textbf{Matrícula:} 0035411\\[5mm]
\egroup

\relatorio{13/08/2020}{31/08/2020}{31/08/2020}
\begin{enumerate}
	\item \atividades Realização do trablho pratico T1 (criação de um script em shell)
	
	\item \aprendizado Trabalhar com descritores de modo a deixar o script como uma execução diferente caso esteja no meio de um pipeLine
	
	\item \duvidas ----
	
	\item \acoes Leitura da documentação passada como orientação para a realização do trabalho.
\end{enumerate}


\relatorio{03/09/2020}{09/09/2020}{09/09/2020}
\begin{enumerate}
	\item \atividades Realização do trablho pratico T2 (PDE)
	
	\item \aprendizado Trabalhar com vetor em Shell
	
	\item \duvidas Como passar um vetor inteiro como argumento para uma função sem ter que transformar em string
	
	\item \acoes Leitura da documentação passada como orientação para a realização do trabalho.
\end{enumerate}


\relatorio{20/10/2020}{29/09/2020}{01/10/2020}
\begin{enumerate}
	\item \atividades Realização do trablho pratico T3 
	
	\item \aprendizado Trabalhar com Matrix
	
	\item \duvidas Sem duvidas
	
	\item \acoes Leitura da documentação passada como orientação para a realização do trabalho e implementação do código.
\end{enumerate}

\relatorio{01/10/2020}{08/10/2020}{08/10/2020}
\begin{enumerate}
	\item \atividades Realização do trablho pratico T4
	
	\item \aprendizado Trabalhar com Thred
	
	\item \duvidas Definição da zona critica
	
	\item \acoes Leitura da documentação passada como orientação para a realização do trabalho e implementação do código.
\end{enumerate}

\relatorio{08/10/2020}{15/10/2020}{15/10/2020}
\begin{enumerate}
	\item \atividades Realização do trablho pratico T5
	
	\item \aprendizado Trabalhar com BD
	
	\item \duvidas Tabela de relacional ( Trabalho com chave estrangeiraaaaa)
	
	\item \acoes Leitura da documentação passada como orientação para a realização do trabalho e implementação do código.
\end{enumerate}

\relatorio{15/10/2020}{22/10/2020}{22/10/2020}
\begin{enumerate}
	\item \atividades Realização do trablho pratico T6
	
	\item \aprendizado Trabalhar com GUI
	
	\item \duvidas Onde Encontrar a documentação para o TKInter
	
	\item \acoes Consulta videos Youtube.
\end{enumerate}

\end{document}
