\documentclass[10pt,a4paper]{report}

\usepackage[brazil]{babel}
\usepackage[utf8]{inputenc}
\usepackage[T1]{fontenc}
\usepackage{graphics,color}
\usepackage[lmargin=3cm,rmargin=2.5cm,tmargin=2cm,bmargin=2.5cm]{geometry}

\newcommand{\bola}{\resizebox{12pt}{!}{$\bullet$\ }}
\newcommand{\relatorio}[3]{\vspace{3mm}\noindent\bola\textbf{Relatório Semana - #1 a #2 - Postado em #3}}
\newcommand{\atividades}{\textbf{Estudos e atividades realizadas:\ }}
\newcommand{\aprendizado}{\textbf{O que aprendi nesta semana:\ }}
\newcommand{\duvidas}{\textbf{Dificuldades e dúvidas que encontrei:\ }}
\newcommand{\acoes}{\textbf{Ações e atitudes para resolver os problemas:\ }}
 
\begin{document}


\begin{center} \LARGE
   \textbf{Modelo Relatório} \\[10mm]
\end{center}

\bgroup \large 
\noindent
\textbf{Disciplina:} Desenvolvimento Linux \\
\textbf{Aluno:} Lucas Mateus Fernandes\\
\textbf{Matrícula:} 0035411\\[5mm]
\egroup

\relatorio{13/08/2020}{31/08/2020}{31/08/2020}
\begin{enumerate}
	\item \atividades Realização do trablho pratico T1 (criação de um script em shell)
	
	\item \aprendizado Trabalhar com descritores de modo a deixar o script como uma execução diferente caso esteja no meio de um pipeLine
	
	\item \duvidas ----
	
	\item \acoes Leitura da documentação passada como orientação para a realização do trabalho.
\end{enumerate}


\relatorio{31/09/2020}{09/09/2020}{09/09/2020}
\begin{enumerate}
	\item \atividades Realização do trablho pratico T2 (PDE)
	
	\item \aprendizado Trabalhar com vetor em Shell
	
	\item \duvidas Como passar um vetor inteiro como argumento para uma função sem ter que transformar em string
	
	\item \acoes Leitura da documentação passada como orientação para a realização do trabalho.
\end{enumerate}



\end{document}
