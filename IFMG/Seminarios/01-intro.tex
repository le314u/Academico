\section{Introdução}

A utilização de descrições informais pode levar a descrições imprecisas e inconsistentes pois diferentes interpretações de uma mesma especificação podem levar facilmente à implementações incompatíveis acarretando em uma consequência desastrosa principalmente quando se trata de sistemas críticos ou de sistemas distribuídos. 
\cite{ufrgs}

Portanto os métodos de especificação formal permitem o desenvolvimento de sistemas sem ambiguidades, através de uma sintaxe e semântica bem definidas.  
\cite{intro}

"Métodos formais" refere-se a técnicas e ferramentas matematicas para a especificação, projeto e verificação de sistemas de software e hardware, pois as ferramentas matematicas usadas em métodos formais são declarações bem formadas em uma lógica matemática e que as verificações formais são deduções rigorosas nessa lógica onde cada etapa segue de uma regra de inferência e, portanto, pode ser verificada por um processo mecânico. Sendo assim os métodos formais fornecem um meio de examinar simbolicamente todo o espaço de estado de um projeto digital (seja hardware ou software) e estabelecer uma propriedade de correção ou segurança que seja verdadeira para todas as entradas possíveis.
\cite{nasa}

Embora o uso da lógica matemática seja um tema unificador na disciplina de métodos formais, não existe um único "método formal" melhor. Cada domínio de aplicativo requer diferentes métodos de modelagem e diferentes abordagens de prova. Além disso, mesmo dentro de um domínio de aplicativo específico, diferentes fases do ciclo de vida podem ser melhor atendidas por diferentes ferramentas e técnicas.
\cite{nasa}

Portanto este trabalho academico tem por finalidade verificar a viabilidade de aplicar a técnica de verificação de modelos em "Pensando em algum software tcc ja feito pelos alunos anteriores"

Talvez o tcc do Diego Maicon Silva

Talvez o tcc do Guilherme Cardoso Silva