\section{Fundamentação Teórica}

\subsection{Model Checking}
Model checking é uma técnica que verifica propriedades de um sistema através de enumeração exaustiva de todos os estados alcançáveis.

Existem duas abordagens para implementar verificação de modelos: a abordagem lógica e a abordagem
que utiliza a teoria dos autômatos.

Na abordagem lógica, um sistema reativo será descrito através de um tipo de
grafo de transição de estados, chamado de estrutura de Kripke, que captura
a intuição do seu comportamento.

Deve-se ainda considerar que existe uma restrição na abordagem lógica: as relações de
transições devem ser totais. As estruturas de Kripke s˜ao simples e suficientes
para capturar os aspectos de comportamento dos sistemas reativos.
[ https://www2.dbd.puc-rio.br/pergamum/tesesabertas/0115648\_03\_cap\_02.pdf ]

\subsubsection{Estrutura de Kripke}
Uma estrutura de Kripke é um conjunto de estados, um conjunto de transições entre estados e uma função que rotula cada estado com o conjunto de propriedades que são verdadeiras nele
Caminhos em estrutura de Kripke são computações em sistemas reativos
[ https://www2.dbd.puc-rio.br/pergamum/tesesabertas/0115648\_03\_cap\_02.pdf ]

\subsection{SubseçãoN}
   