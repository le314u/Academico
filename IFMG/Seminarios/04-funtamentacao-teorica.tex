
[https://www.maxwell.vrac.puc-rio.br/10082/10082\_4.PDF]   

\section{Fundamentação Teórica}

\subsection{Model Checking}
Model checking é uma técnica que consiste na verificação automática de propriedades acerca do comportamento de sistemas através de enumeração exaustivade todos os estados alcançáveis. 

Para realizar a validação das propriedades de sistemas reativos é necessário:
Especificar quais as propriedades que o sistema deverá ter para que seja considerado correto; Construir um modelo formal do sistema de modo que capture todas as propriedades essenciais do sistema; Executar o verificador de modelos para validar as propriedades especificadas do sistema.

Caso todas as propriedades sejam verdadeiras, então o sistema está correto. Caso não obedeça a alguma propriedade, então é dado um contra exemplo mostrando o porquê da não verificação da propriedade.

[ https://www2.dbd.puc-rio.br/pergamum/tesesabertas/0115648\_03\_cap\_02.pdf ]



\subsubsection{Sistemas Reativos}
Sistemas reativos têm como caracterização básica a computação.

Para efeitos de definição:Estado é a descrição do sistema em um dado instante de tempo; Transição é uma relação entre dois estados; 
Computação é uma sequência de estados onde cada estado é obtido através de um estado anterior e uma relação de transição entre eles.
[ https://www2.dbd.puc-rio.br/pergamum/tesesabertas/0115648\_03\_cap\_02.pdf ]

\subsubsection{Estrutura de Kripke}

Um sistema reativo pode ser descrito através de uma estrutura de Kripke pois caminhos em estrutura de Kripke são computações em sistemas reativos.

Uma estrutura de Kripke é uma quadrupla que contem um conjunto de estados S, conjunto de estados inicias Si que é subconjunnto de S, um conjunto de transições entre estados R e uma função L que rotula cada estado com o conjunto de propriedades que são verdadeiras nele.

[ https://www2.dbd.puc-rio.br/pergamum/tesesabertas/0115648\_03\_cap\_02.pdf ]

[https://www.maxwell.vrac.puc-rio.br/10082/10082\_4.PDF]