\documentclass[
  12pt,
  arial,
  openany,
  oneside,
  chapter=TITLE,
  section=TITLE,
  subsection=TITLE,
  subsubsection=TITLE,
  a4paper,
  table,
  english,
  brazil]{abntex2}

% packages
\usepackage{lmodern}
\usepackage[T1]{fontenc}
\usepackage[utf8]{inputenc}
\usepackage{lastpage}
\usepackage{pgfplots}
\usepackage{indentfirst}
\usepackage{color}
\usepackage{graphicx}
\usepackage{microtype}
\renewcommand\thesection{\arabic{section}}
\usepackage[brazilian,hyperpageref]{backref}
\usepackage[alf]{abntex2cite}

\titulo{Analise de viabilidade de aplicação do metodo formal Model Cheking no trabalho apresentado por 'x'}
\autor{Lucas Mateus Fernandes}
\local{Formiga - MG}
\data{2020}
\orientador{Mário Luiz Rodrigues Oliveira}
\instituicao{%
 Instituto Federal de Educação, Ciência e Tecnologia de Minas Gerais \par
 Campus Formiga \par
 Ciência da Computação
}

\preambulo{Proposta de projeto do Trabalho de Conclusão de Curso do Curso de Bacharelado em Ciência da Computação apresentada ao IFMG - Campus Formiga.}

\begin{document}
	\selectlanguage{brazil}
	\frenchspacing
	
	\bibliography{referencias}
	\bibliographystyle{ieeetr}

	\imprimircapa
	
	\imprimirfolhaderosto
	
	\newpage
	
	\tableofcontents
	
	\newpage
	\section{Introdução}

A utilização de descrições informais pode levar a descrições imprecisas e inconsistentes pois diferentes interpretações de uma mesma especificação podem levar facilmente à implementações incompatíveis acarretando em uma consequência desastrosa principalmente quando se trata de sistemas críticos ou de sistemas distribuídos. 
[ http://www.lasid.ufba.br/projetos/metodos.html ] [ https://www.inf.ufrgs.br/site/pesquisa/grupos-de-pesquisa/fundamentos-da-computacao-e-metodos-formais/ ]
\cite{teste}
Portanto os métodos de especificação formal permitem o desenvolvimento de sistemas sem ambiguidades, através de uma sintaxe e semântica bem definidas.  
[file:///home/guest/Downloads/tut-met-formais.pdf]
\cite{teste}
"Métodos formais" refere-se a técnicas e ferramentas matematicas para a especificação, projeto e verificação de sistemas de software e hardware, pois as ferramentas matematicas usadas em métodos formais são declarações bem formadas em uma lógica matemática e que as verificações formais são deduções rigorosas nessa lógica onde cada etapa segue de uma regra de inferência e, portanto, pode ser verificada por um processo mecânico. Sendo assim os métodos formais fornecem um meio de examinar simbolicamente todo o espaço de estado de um projeto digital (seja hardware ou software) e estabelecer uma propriedade de correção ou segurança que seja verdadeira para todas as entradas possíveis.
[https://shemesh.larc.nasa.gov/fm/fm-what.html]
\cite{teste}
Embora o uso da lógica matemática seja um tema unificador na disciplina de métodos formais, não existe um único "método formal" melhor. Cada domínio de aplicativo requer diferentes métodos de modelagem e diferentes abordagens de prova. Além disso, mesmo dentro de um domínio de aplicativo específico, diferentes fases do ciclo de vida podem ser melhor atendidas por diferentes ferramentas e técnicas.
[https://shemesh.larc.nasa.gov/fm/fm-what.html]
\cite{teste}
	\section{Justificativa}
Atualmente, a maioria das empresas focam seus recursos em entregar o produto com agilidade e não enfatizam o fato de garantir que o produto funcione com exatidão pois conforme uma máxima do capistalimos vulgarmente utilizada “Tempo é dinheiro" porem nada adianta um software que não faz o que deveria fazer.

Portanto este Trabalho de conclusão de curso se justifica pelo fato de poder auxiliar na escolha de um metodo formal a ser aplicado a fim de garantir a corretude do software.


	\section{Objetivos}

\subsection{Objetivo Geral}

O objetivo geral deste trabalho é definir entre 2 metodos formais: Model Checking e Cálculo de Processos, qual o melhor a ser aplicado em "ESCOLHER O SOFTWARE"


\subsection{Objetivos Específicos}
\begin{enumerate}
    \item Fazer um levantamento bibliografico sobre CSP
    \item Fazer um levantamento bibliografico sobre Model Cheking
    \item Elencar as Principais diferenças entre CSP e Model Checking
    \item Definir se CSP ou Model Checking é melhor para o caso de uso definido
\end{enumerate}




	\section{Fundamentação Teórica}

\subsection{Model Checking}
Model checking é uma técnica que verifica propriedades de um sistema através de enumeração exaustiva de todos os estados alcançáveis.

Existem duas abordagens para implementar verificação de modelos: a abordagem lógica e a abordagem
que utiliza a teoria dos autômatos.

Na abordagem lógica, um sistema reativo será descrito através de um tipo de
grafo de transição de estados, chamado de estrutura de Kripke, que captura
a intuição do seu comportamento.

Deve-se ainda considerar que existe uma restrição na abordagem lógica: as relações de
transições devem ser totais. As estruturas de Kripke s˜ao simples e suficientes
para capturar os aspectos de comportamento dos sistemas reativos.
[ https://www2.dbd.puc-rio.br/pergamum/tesesabertas/0115648\_03\_cap\_02.pdf ]

\subsubsection{Estrutura de Kripke}
Uma estrutura de Kripke é um conjunto de estados, um conjunto de transições entre estados e uma função que rotula cada estado com o conjunto de propriedades que são verdadeiras nele
Caminhos em estrutura de Kripke são computações em sistemas reativos
[ https://www2.dbd.puc-rio.br/pergamum/tesesabertas/0115648\_03\_cap\_02.pdf ]

\subsection{SubseçãoN}
   
	\section{Metodologia}


\newpage
	\section{Cronograma}

% Please add the following required packages to your document preamble:
% \usepackage[table,xcdraw]{xcolor}
% If you use beamer only pass "xcolor=table" option, i.e. \documentclass[xcolor=table]{beamer}
\begin{table}[h]
\begin{tabular}{|l|l|l|l|l|l|l|l|l|}
\hline
ATIVIDADES  & MÊS1  & MÊS2  & MÊS3  & MÊS4  & MÊS5  & MÊS6  & MÊS7  & MÊS8  \\ \hline
\begin{tabular}[c]{@{}l@{}}Levantamento Bibliografico \end{tabular} & \cellcolor[HTML]{656565} &    &   &   &   &   &   &   \\ \hline
\begin{tabular}[c]{@{}l@{}}Elaboração da Tabela \end{tabular}   &   & \cellcolor[HTML]{656565}  &   &   &   &   &   &   \\ \hline
\begin{tabular}[c]{@{}l@{}}Estudo do software X \end{tabular}   &   &   & \cellcolor[HTML]{656565}& \cellcolor[HTML]{656565} & \cellcolor[HTML]{656565} &   &    &  \\ \hline
\begin{tabular}[c]{@{}l@{}}Preenchimento da tabela \end{tabular}    &   &   &   &   & \cellcolor[HTML]{656565} & \cellcolor[HTML]{656565} &  \cellcolor[HTML]{656565}   &   \\ \hline
\begin{tabular}[c]{@{}l@{}}Apresentação pública \\ de TCC\end{tabular}  &   &   &   &   &   &   &   & \cellcolor[HTML]{656565} \\ \hline
\end{tabular}
\end{table}
	\section{Resultados Esperados}

\end{document}