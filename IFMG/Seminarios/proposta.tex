\documentclass[
  12pt,
  arial,
  openany,
  oneside,
  chapter=TITLE,
  section=TITLE,
  subsection=TITLE,
  subsubsection=TITLE,
  a4paper,
  table,
  english,
  brazil]{abntex2}

% packages
\usepackage{lmodern}
\usepackage[T1]{fontenc}
\usepackage[utf8]{inputenc}
\usepackage{lastpage}
\usepackage{pgfplots}
\usepackage{indentfirst}
\usepackage{color}
\usepackage{graphicx}
\usepackage{microtype}
\renewcommand\thesection{\arabic{section}}
\usepackage[brazilian,hyperpageref]{backref}
\usepackage[alf]{abntex2cite}

\titulo{Analise de viabilidade de aplicação do metodo formal Model Cheking no trabalho apresentado por 'x'}
\autor{Lucas Mateus Fernandes}
\local{Formiga - MG}
\data{2020}
\orientador{Mário Luiz Rodrigues Oliveira}
\instituicao{%
 Instituto Federal de Educação, Ciência e Tecnologia de Minas Gerais \par
 Campus Formiga \par
 Ciência da Computação
}

\preambulo{Proposta de projeto do Trabalho de Conclusão de Curso do Curso de Bacharelado em Ciência da Computação apresentada ao IFMG - Campus Formiga.}

\begin{document}
	\selectlanguage{brazil}
	\frenchspacing
	
	\bibliography{referencias}
	\bibliographystyle{ieeetr}

	\imprimircapa
	
	\imprimirfolhaderosto
	
	\newpage
	
	\tableofcontents
	
	\newpage
	\section{Introdução}

A utilização de descrições informais pode levar a descrições imprecisas e inconsistentes pois diferentes interpretações de uma mesma especificação podem levar facilmente à implementações incompatíveis acarretando em uma consequência desastrosa principalmente quando se trata de sistemas críticos ou de sistemas distribuídos. 
[ http://www.lasid.ufba.br/projetos/metodos.html ] [ https://www.inf.ufrgs.br/site/pesquisa/grupos-de-pesquisa/fundamentos-da-computacao-e-metodos-formais/ ]
\cite{teste}
Portanto os métodos de especificação formal permitem o desenvolvimento de sistemas sem ambiguidades, através de uma sintaxe e semântica bem definidas.  
[file:///home/guest/Downloads/tut-met-formais.pdf]
\cite{teste}
"Métodos formais" refere-se a técnicas e ferramentas matematicas para a especificação, projeto e verificação de sistemas de software e hardware, pois as ferramentas matematicas usadas em métodos formais são declarações bem formadas em uma lógica matemática e que as verificações formais são deduções rigorosas nessa lógica onde cada etapa segue de uma regra de inferência e, portanto, pode ser verificada por um processo mecânico. Sendo assim os métodos formais fornecem um meio de examinar simbolicamente todo o espaço de estado de um projeto digital (seja hardware ou software) e estabelecer uma propriedade de correção ou segurança que seja verdadeira para todas as entradas possíveis.
[https://shemesh.larc.nasa.gov/fm/fm-what.html]
\cite{teste}
Embora o uso da lógica matemática seja um tema unificador na disciplina de métodos formais, não existe um único "método formal" melhor. Cada domínio de aplicativo requer diferentes métodos de modelagem e diferentes abordagens de prova. Além disso, mesmo dentro de um domínio de aplicativo específico, diferentes fases do ciclo de vida podem ser melhor atendidas por diferentes ferramentas e técnicas.
[https://shemesh.larc.nasa.gov/fm/fm-what.html]
\cite{teste}
	\section{Justificativa}
Atualmente, a maioria das empresas focam seus recursos em entregar o produto com agilidade e não enfatizam o fato de garantir que o produto funcione com exatidão pois conforme uma máxima do capistalimos vulgarmente utilizada “Tempo é dinheiro" porem nada adianta um software que não faz o que deveria fazer.

Portanto este Trabalho de conclusão de curso se justifica pelo fato de poder auxiliar na escolha de um metodo formal a ser aplicado a fim de garantir a corretude do software.
	\section{Objetivos}

\subsection{Objetivo Geral}

O objetivo geral deste trabalho é elencar as principais diferenças entre 'n' metodos formais
e verificar a viabilidade de cada metodo tendo como escopo de uso 

\subsection{Objetivos Específicos}
\begin{enumerate}
    \item objetivo específico1
    \item objetivo específico2
    \item objetivo específico3
    \item objetivo específico4
    \item objetivo específicoN
\end{enumerate}




	
[https://www.maxwell.vrac.puc-rio.br/10082/10082\_4.PDF]   

\section{Fundamentação Teórica}

\subsection{Model Checking}
Model checking é uma técnica que consiste na verificação automática de propriedades acerca do comportamento de sistemas através de enumeração exaustivade todos os estados alcançáveis. 

Para realizar a validação das propriedades de sistemas reativos é necessário:
Especificar quais as propriedades que o sistema deverá ter para que seja considerado correto; Construir um modelo formal do sistema de modo que capture todas as propriedades essenciais do sistema; Executar o verificador de modelos para validar as propriedades especificadas do sistema.

Caso todas as propriedades sejam verdadeiras, então o sistema está correto. Caso não obedeça a alguma propriedade, então é dado um contra exemplo mostrando o porquê da não verificação da propriedade.

[ https://www2.dbd.puc-rio.br/pergamum/tesesabertas/0115648\_03\_cap\_02.pdf ]



\subsubsection{Sistemas Reativos}
Sistemas reativos têm como caracterização básica a computação.

Para efeitos de definição:Estado é a descrição do sistema em um dado instante de tempo; Transição é uma relação entre dois estados; 
Computação é uma sequência de estados onde cada estado é obtido através de um estado anterior e uma relação de transição entre eles.
[ https://www2.dbd.puc-rio.br/pergamum/tesesabertas/0115648\_03\_cap\_02.pdf ]

\subsubsection{Estrutura de Kripke}

Um sistema reativo pode ser descrito através de uma estrutura de Kripke pois caminhos em estrutura de Kripke são computações em sistemas reativos.

Uma estrutura de Kripke é uma quadrupla que contem um conjunto de estados S, conjunto de estados inicias Si que é subconjunnto de S, um conjunto de transições entre estados R e uma função L que rotula cada estado com o conjunto de propriedades que são verdadeiras nele.

[ https://www2.dbd.puc-rio.br/pergamum/tesesabertas/0115648\_03\_cap\_02.pdf ]

[https://www.maxwell.vrac.puc-rio.br/10082/10082\_4.PDF]
	\section{Metodologia}

Será elaborada e realizada uma revisão sistemática sobre o model checking.
Após o levantamento bibliografico será criada uma tabela de carater quantitativo sobre a viabilidade de aplicação do metodo formal em um software.
Com a tabela pronta será realizada uma analise do software 'X' e definida sua viabilidade de acordo com a tabela elaborada.

\newpage
	\section{Cronograma}

% Please add the following required packages to your document preamble:
% \usepackage[table,xcdraw]{xcolor}
% If you use beamer only pass "xcolor=table" option, i.e. \documentclass[xcolor=table]{beamer}
\begin{table}[h]
\begin{tabular}{|l|l|l|l|l|l|l|l|l|}
\hline
ATIVIDADES  & MÊS1  & MÊS2  & MÊS3  & MÊS4  & MÊS5  & MÊS6  & MÊS7  & MÊS8  \\ \hline
\begin{tabular}[c]{@{}l@{}}Levantamento Bibliografico \end{tabular} & \cellcolor[HTML]{656565} &    &   &   &   &   &   &   \\ \hline
\begin{tabular}[c]{@{}l@{}}Elaboração da Tabela \end{tabular}   &   & \cellcolor[HTML]{656565}  &   &   &   &   &   &   \\ \hline
\begin{tabular}[c]{@{}l@{}}Estudo do software X \end{tabular}   &   &   & \cellcolor[HTML]{656565}& \cellcolor[HTML]{656565} & \cellcolor[HTML]{656565} &   &    &  \\ \hline
\begin{tabular}[c]{@{}l@{}}Preenchimento da tabela \end{tabular}    &   &   &   &   & \cellcolor[HTML]{656565} & \cellcolor[HTML]{656565} &  \cellcolor[HTML]{656565}   &   \\ \hline
\begin{tabular}[c]{@{}l@{}}Apresentação pública \\ de TCC\end{tabular}  &   &   &   &   &   &   &   & \cellcolor[HTML]{656565} \\ \hline
\end{tabular}
\end{table}
	\section{Resultados Esperados}
Após o desenvolvimento deste trabalho, é esperado uma resposta objetiva sobre a viabilidade de aplicação do metodo formal no software X.
E um ponto de partida para trabalhos seguintes que desejem aplica o metodo formal em software semelhantes.

\end{document}