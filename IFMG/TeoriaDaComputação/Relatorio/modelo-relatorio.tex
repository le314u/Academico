\documentclass[10pt,a4paper]{report}

\usepackage[brazil]{babel}
\usepackage[utf8]{inputenc}
\usepackage[T1]{fontenc}
\usepackage{graphics,color}
\usepackage[lmargin=3cm,rmargin=2.5cm,tmargin=2cm,bmargin=2.5cm]{geometry}

\newcommand{\bola}{\resizebox{12pt}{!}{$\bullet$\ }}
\newcommand{\relatorio}[3]{\vspace{3mm}\noindent\bola\textbf{Relatório Semana - #1 a #2 - Postado em #3}}
\newcommand{\atividades}{\textbf{Estudos e atividades realizadas:\ }}
\newcommand{\aprendizado}{\textbf{O que aprendi nesta semana:\ }}
\newcommand{\duvidas}{\textbf{Dificuldades e dúvidas que encontrei:\ }}
\newcommand{\acoes}{\textbf{Ações e atitudes para resolver os problemas:\ }}
 
\begin{document}


\begin{center} \LARGE
   \textbf{Modelo Relatório} \\[10mm]
\end{center}

\bgroup \large 
\noindent
\textbf{Disciplina:} Teoria da Computação \\
\textbf{Aluno:} Lucas Mateus Fernandes\\
\textbf{Matrícula:} 0035411\\[5mm]
\egroup

\relatorio{23/08/2020}{31/08/2020}{31/08/2020}
\begin{enumerate}
	\item \atividades Realização do trablho pratico T1 (criação da maquina de turing com açucar sintático)
	
	\item \aprendizado ----
	
	\item \duvidas ----
	
	\item \acoes Leitura da documentação passada como orientação para a realização do trabalho, Utilização de ferramentas como Discord e Google Meet para fazer video chamadas com compartilhamento de tela para a execução do trabalho.
\end{enumerate}


\relatorio{31/09/2020}{15/09/2020}{15/09/2020}
\begin{enumerate}
	\item \atividades Realização do trablho pratico T2 (programar a mt para fazer soma e subtração de dois elementos)
	
	\item \aprendizado Processo de soma e subtração levando em consideração o valor posicional dos digitos (É algo tão automático que não percebia o processo)
	
	\item \duvidas 'Como fazer uma subtração A-B sem comutar a ordem dos elementos sendo que |B| > |A|, de todas as maneiras que pensei não tive exito, tive que fazer -1 * ( A - B )'
	
	\item \acoes Leitura da documentação passada como orientação para a realização do trabalho.
\end{enumerate}



\end{document}
